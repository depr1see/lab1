
\documentclass[12pt, a4paper]{article}
\usepackage[utf8]{inputenc}
\usepackage{multicol} %колонки
\usepackage{setspace} %межстрочный интервал
\usepackage{ragged2e}% выравнивания текста по ширине в документе.
\usepackage{fancyhdr} %настройки верхнего и нижнего колонтитулов в документе.
\usepackage{titlesec} %стилей заголовков разделов в документе.
\usepackage{enumitem} %настройки списков в документе.
\usepackage{lipsum}
\usepackage[T2A]{fontenc}
\usepackage{graphicx}%Вставка картинок правильная
\usepackage{float}%"Плавающие" картинки
\usepackage{wrapfig}%Обтекание фигур (таблиц, картинок и прочего)
\usepackage[labelformat=empty]{caption}
\usepackage[left=2.5cm,right=2.4cm, top=2.2cm,bottom=2.5cm]{geometry}
\usepackage{etoolbox} % for '\patchcmd' macro
\graphicspath{ {images/} }
\justifying % выравнивает текст по ширине.
\fancyhf{} %очищает все верхние и нижние колонтитулы.
\renewcommand{\headrulewidth}{0pt} 
\cfoot{\vskip -1.5cm \thepage} %устанавливает номер страницы в нижнем колонтитуле.
\linespread{0.7} %устанавливает межстрочный интервал 

\setlength{\columnsep}{0.65cm}
\setcounter{page}{21}
\renewcommand{\thesection}{\Roman{section}} %устанавливает стиль нумерации разделов в виде заглавных римских цифр.
\setcounter{figure}{1}
\titleformat{\section}{\footnotesize\centering\sc}{\thesection.}{0cm}{}[] %настраивает стиль заголовков разделов.

\begin{document}


\fontsize{10}{15}\selectfont
\begin{multicols}{2}
\setlength{\parindent}{0.8cm}
\setlength{\parindent}{0.4cm}
\fontsize{10}{15}\selectfont

\begin{itemize}
\setlength{\itemsep}{0pt}
  \setlength{\parskip}{1pt}
  \setlength{\parsep}{0pt}
\item[]  The hierarchical nature of the agent system makes it
easy to develop and modify such systems by analogy
with the hierarchical structure of problem solvers in
\textit{individual ostis-systems} [4], [11]
\item   In traditional systems, often all agents of the system,
or at least a significant part of them, may be
involved in problem solving. Taking into account the
complexity of ostis-systems included in the OSTIS
Ecosystem, such a situation is unlikely in the OSTIS
Ecosystem and most often in the near future several
ostis-systems, most often belonging to one ostis-community, will be involved in problem solving.
\item  Traditional self-organizing systems are usually considered in isolation from the means of representation of information processed in such systems, i.e. neither the form of representation of processed
information, nor the semantics of processed information are explicitly fixed. An important advantage of OSTIS \textit{Ecosystem} and OSTIS \textit{Technology} as a whole is the orientation on unified and universal
models of information representation, realized in the form of OSTIS \textit {Technology} and a family of top-level ontologies built on its basis. This approach allows us to say
\begin{enumerate}[label=\textbf{--},leftmargin=*]   
\setlength{\itemsep}{0pt}
  \setlength{\parskip}{1pt}
  \setlength{\parsep}{0pt}
\item about universality of the developed mechanisms
of collective problem solving within OSTIS \textit {Ecosystem}, i.e. the possibility to unlimitedly increase the possibilities of OSTIS \textit {Ecosystem} to
automate different kinds of human activities in various fields;
\item about the possibility to describe the agents of
the OSTIS Ecosystem by the same means that
are used to describe the information processed
by the agents, with the required degree of detail. Thus, it becomes possible to analyze the
specification of some agents (e.g., their functional
capabilities, classes of solved tasks, etc.) by other
agents, which opens new opportunities for selforganization in collective problem solving.
\item about the possibility of exchanging information
constructions of arbitrary complexity between
agents, there is no need to limit the possible
semantics of such messages and, moreover, to fix
their structure, as it is often done in traditional
approaches. It should be emphasized that agents
in the framework of the proposed approach do not
exchange messages directly, but specify in a common knowledge base the actions they perform,
so there are no fundamental restrictions on the
content of such specification
\end{enumerate}
\end{itemize}\par
The analysis of the presented features allows us to draw
the following conclusions:
\begin{itemize}
\setlength{\itemsep}{0pt}
  \setlength{\parskip}{1pt}
  \setlength{\parsep}{0pt}
\item Direct application of existing approaches to selforganization in multi-agent systems for solving OSTIS \textit {Ecosystem} problems is not possible due to its
essential features, but many known approaches can be adapted for solving a number of specific problems, for example, when organizing the exchange of messages between ostis-systems at the physical level, searching for the most suitable executor for solving this or that task and so on;
\item At the same time, OSTIS \textit {Ecosystem} features allow
not to consider at the level of collective problem
solving a number of non-trivial problems related to
reliability assurance and optimization of load distribution between ostis-systems, and open new opportunities to expand the range of possible spheres of
human activity automation, to ensure interoperability of corresponding intelligent systems and their collectives and to reduce the labor intensity of their maintenance and evolution.
\end{itemize}\par
IV. Proposed approach to problem solving within the
next-generation intelligent computer systems ecosystem\par
The key difference of the proposed approach to the
organization of decentralized problem solving in OSTIS
Ecosystem in contrast to traditional approaches to self-organization is that within OSTIS \textit {Ecosystem} we initially
in the process of ecosystem development form a hierarchy
of ostis-systems and their specification so as to further
simplify the process of organizing collective problem
solving, in particular, the formation of a collective of
performers, the transfer of messages between performers,
etc., leaving the opportunity for continuous refinement
and improvement. Thus, the agent system is initially built
in such a way as to make self-organization more convenient by analogy with the way top-level ontologies are
developed to ensure compatibility of ontologies instead
of developing ontologies independently of each other and
then solving the problem of ontology alignment, which
most often turns out to be not trivial at all.\par
In the paper [14] a methodology for the design of
multi-agent systems including five stages is proposed.
Let us consider in more detail the application of this
methodology in the context of problem solving within
the OSTIS \textit {Ecosystem}:
\begin{itemize}
\setlength{\itemsep}{0pt}
  \setlength{\parskip}{1pt}
  \setlength{\parsep}{0pt}
    \item \textbf {Step 1: Analyze the objectives for which the
system is being developed.} The purpose of the
work in this case is to ensure the potential possibility of solving any problems arising within the
OSTIS \textit {Ecosystem}, which requires the availability
of universal and unified means of describing the
goals and objectives. Within the OSTIS \textit {Technology}
framework, common unified means of specification
of actions and tasks are proposed [4]. As far as
semantic completeness of such tools is concerned
(taking into account all possible classes of tasks that
may arise), it is proposed to take as a basis the task
ontology proposed in [15]
\item  \textbf {Step 2: Designing the overall structure of the
multi-agent system.} Within the framework of the
considered approach, the structure of the multiagent system is based on the architecture of the
OSTIS \textit {Ecosystem} discussed above and is constantly
refined taking into account the addition of new
agents to the OSTIS \textit {Ecosystem}. In terms of classification of agents of OSTIS \textit {Ecosystem} at the current
stage it is proposed to single out only \textit {corporate
ostis-systems} into a separate class due to the fact
that they play a special role in the process of organizing collective problem solving. The principles of
agents’ communication via \textit {corporate ostis-systems}
are discussed in more detail at Step 4.
\item \textbf{3: Designing the internal structure of the
agent and the principles of its operation.} Since
all OSTIS \textit{Ecosystem} agents are ostis-systems (even
users of the OSTIS Ecosystem interact with it
through personal ostis-assistants, which are ostissystems [4], [8]), additional specification of the
principles of their structure is not required, as it
is discussed in detail in the works devoted to the
OSTIS Technology [4], [16]. To ensure the possibility of interaction between ostis-systems over
the network, it is proposed to add an interface
subsystem to each system, which is discussed in
more detail in Step 5.
\item \textbf{Step 4: Develop the principles of agent interaction.} As mentioned earlier, it is proposed to base the
principles of agents’ communication within OSTIS
\textit{Ecosystem} during collective problem solving on the
principles of agents’ communication in the memory
of ostis-systems (sc-agents). In the work [7] an
approach is proposed assuming that one of the ostissystems included in the collective of ostis-systems
will be used as a tool of communication for the
participants of the collective of ostis-systems. If
such collective is formed on a permanent basis (is
a \textit{ostis-community} or a part of it), it is proposed
to use the \textit{corporate ostis-system} of the specified
ostis-community as such communicator system. If
a collective of ostis-systems is formed temporarily
for solving one or several complex problems, i.e.
it is necessary to temporarily involve \tetxtit{ostis-systems}
belonging to several \textit{ostis-communities}, two variants
of organizing communication of ostis-systems are
possible:
\begin{enumerate}[label=\textbf{--},leftmargin=*]
\setlength{\itemsep}{0pt}
  \setlength{\parskip}{1pt}
  \setlength{\parsep}{0pt}
    \item One of the systems belonging to such a temporary collective of ostis-systems is selected as
a communicator system. In this case, such an
ostis-system becomes temporarily the \textit{corporate
ostis-system} of the temporary \textit{ostis-community}.
Accordingly, in this case it is required to install in
the ostis-system an interface subsystem for ostis-systems and to load into its knowledge base the
specifications of other ostis-systems participating
in the problem solving process. Thus, the cost of
preliminary preparation of a collective of ostis-systems for problem solving can be quite serious,
and this approach may be ineffective for relatively
simple problems solving.
\item The \textit{corporate ostis-system} of the closest hierarchical \textit{ostis-community} is chosen as the communicator system, such that all ostis-systems required for the solution belong either to this ostiscommunity or to more private ostis-communities
(possibly on several hierarchical levels). In the
example of the \textit{ostis-communities} hierarchy fragment shown in Figure 2 , assuming that the problem solving requires the participation of \textit{ostis-systems} OS1, OS2, and OS3, then the \textit{corporate
ostis-system} of \textit{ostis-community} OC1 will be selected as the communicator system.
\begin{figure}[H]
    \centering
    \includegraphics[width=8.5cm]{ЛабаН1/F1.png}
    \caption {\fontsize{8}{7}\selectfont Figure 2: Example of communicator system selection}
    
\end{figure}
According to the above architecture of the OSTIS
Ecosystem such an ostis-community will always
exist, in the extreme case the role of such a
corporate system will be played by the OSTIS
\textit{Metasystem}. The disadvantage of this communication option is that sending messages between
the participants of the problem solving process
may generally take more time due to the increased
path between the corporate ostis-system and the
ostis-systems which are performers.
\end{enumerate}
It is important to note that in the presence of
such a communicator system, agents at the logical
level do not exchange messages directly, but communicate by specifying their actions in the shared
memory of the communicator system; nevertheless,
at the physical level, messages are forwarded between ostis-systems, generally physically located in
different nodes of the network, arbitrarily distant
from each other. This idea of separating the logical
and physical layers of communication is realized
within the concept of \textit{overlay networks} [17]. An
overlay network is a virtual network whose structure
differs from the real communication network on
the basis of which this overlay network functions.
The idea of using \textit{corporate ostis-systems} as a basis
for agents’ communication in such a network and
a repository of agents’ specifications and methods
provided by them can be considered as a new stage
of development of the idea of P2P agent platform
developed by the FIPA consortium [17]. The main
idea of such a platform is to provide all agents in the
network with the possibility of semantic search for
the services they need, as well as to search for agents
that possess the required services. Another function
of the platform is to support transparent communication between agents-customers and agents-owners
of services [17]. In general, an agent network may
have several such platforms, each responsible for
a different segment of the network, similar to the
way a \textit{corporate ostis-system} is responsible for a
corresponding \textit{ostis-community}.\\
To implement the language of interaction between
ostis-systems, it is proposed to use the ideas of
wave programming [18], [19], as well as insertion
programming [20], [21] as a basis. Later variants of
the wave language theory development were called
Spatial Grasp Technology [18], [19], within which
Spatial Grasp Language is developed accordingly.
Implementing such interaction requires the development of at least two levels of languages:
\begin{enumerate}[label=\textbf{--},leftmargin=*] 
\setlength{\itemsep}{0pt}
  \setlength{\parskip}{1pt}
  \setlength{\parsep}{0pt}
    \item transport layer defining the principles of recording SC-code constructions in some format convenient for network transmission. As a variant of such language it is proposed to use SCs-code [4];
 \item semantic level defining the content of messages transmitted over the network. The SCP language, which is the basic programming language for ostis-systems, is proposed to be used as a basis for such a language [4].
\end{enumerate}
It is important to note that within the framework of
the proposed approach, the \textit{corporate ostis-system}
acts as a common information resource and notifies
the participants of the problem solving process
about the events, but does not control the problem
solving process directly. Thus, it is not a question of
rigid \underline{imperative management}, but of more flexible
\underline{declarative}. This allows us to realize the advantages
of the sc-agent interaction mechanism in a shared
semantic memory [4], [11], such as modifiability
of the agent system, convenience of its design and
others.
\item \textbf{Step 5: Develop the detailed architecture of the
multi-agent system.} At this stage, it is supposed to
clarify the principles of interaction between agents
and the environment, taking into account the previously clarified agent structure and the principles of
their interaction.\\
Implementation of the proposed approach assumes
that each ostis-system will include a communication
interface subsystem that will receive messages from
the external environment (from the \textit{corporate ostis=system}), transform them into tasks for internal scagents of the \textit{ostis-system}, and then transform the
result of their work into a response message and
send it to the corresponding recipient. An important
feature of such a subsystem is that it behaves as
a sc-agent when interacting with internal sc-agents
and communicates with other sc-agents according
to the same principles. This allows to ensure the
independence of the development and evolution of
such a subsystem from other components of the
ostis-system and to exclude the necessity to take
into account the fact of future interaction of the
ostis-system with other ostis-systems at the stage of
its design. In other words, an ostis-system solves
a subtask within a distributed collective of ostissystems just as if it were solving a problem explicitly
formulated, for example, by a user. This approach
greatly simplifies the design of ostis-systems collectives, eliminating explicit dependencies between
them and the need to provide for the necessity of
collective problem solving in advance.\\
In turn, each \textit{corporate ostis-system}, when interpreting a particular method, interacts with other
ostis-systems as if they were internal sc-agents of
this ostis-system. Thus, each \textit{corporate ostis-system}
includes an interface subsystem that converts events
in the memory of the corporate \textit{ostis-system} into
messages to other \textit{ostis-system}s and response messages from these \textit{ostis-systems} into corresponding
information constructs in the knowledge base of
the \textit{corporate ostis-system}. This approach allows to
ensure the independence of \textit{corporate ostis-systems}
from other ostis-systems participating in the problem solving process and to exclude the necessity to
provide in advance the necessity of collective problem solving not only when designing conventional
ostis-systems, but also when designing \textit{corporate
ostis-systems}. An illustration of this approach will
be given below.
\end{itemize}\par
From the point of view of the modern classification
of self-organization methods in multi-agent systems [17],
the proposed approach of agent interaction at the logical
level can be considered as a kind of mechanism based
on indirect interactions of organizational agents. Mecha-

\end{multicols}
\end{document}